%%%%%%%%%%%%%%%%%%%%%%%%%%%%%%% beamer %%%%%%%%%%%%%%%%%%%%%%%%%%%%%%%%%%%%%%%%%%%%%%%%%
% To run - pdflatex filename.tex
%	   acroread filename.pdf
%%%%%%%%%%%%%%%%%%%%%%%%%%%%%%%%%%%%%%%%%%%%%%%%%%%%%%%%%%%%%%%%%%%%%%%%%%%%%%%%%%%%%%%%
% include packages


\documentclass[compress,10pt]{beamer}

\mode<presentation>

\usetheme{Dresden}

% other themes: AnnArbor, Antibes, Bergen, Berkeley, Berlin, Boadilla, boxes, CambridgeUS, Copenhagen, Darmstadt, default, Dresden, Frankfurt, Goettingen,
% Hannover, Ilmenau, JuanLesPins, Luebeck, Madrid, Maloe, Marburg, Montpellier, PaloAlto, Pittsburg, Rochester, Singapore, Szeged, Warsaw classic

%\usecolortheme{lily}
% color themes: albatross, beaver, beetle, crane, default, dolphin, dov, fly, lily, orchid, rose, seagull, seahorse, sidebartab, structure, whale, wolverine

%\usefonttheme{serif}
% font themes: default, professionalfonts, serif, structurebold, structureitalicserif, structuresmallcapsserif

\hypersetup{pdfpagemode=FullScreen} % makes your presentation go automatically to full screen

% define your own colors:
\definecolor{Red}{rgb}{1,0,0}
\definecolor{Blue}{rgb}{0,0,1}
\definecolor{Green}{rgb}{0,1,0}
\definecolor{magenta}{rgb}{1,0,.6}
\definecolor{lightblue}{rgb}{0,.5,1}
\definecolor{lightpurple}{rgb}{.6,.4,1}
\definecolor{gold}{rgb}{.6,.5,0}
\definecolor{orange}{rgb}{1,0.4,0}
\definecolor{hotpink}{rgb}{1,0,0.5}
\definecolor{newcolor2}{rgb}{.5,.3,.5}
\definecolor{newcolor}{rgb}{0,.3,1}
\definecolor{newcolor3}{rgb}{1,0,.35}
\definecolor{darkgreen1}{rgb}{0, .35, 0}
\definecolor{darkblue}{rgb}{0,0,.50}
\definecolor{darkgreen}{rgb}{0, .6, 0}
\definecolor{darkred}{rgb}{.75,0,0}

\xdefinecolor{olive}{cmyk}{0.64,0,0.95,0.4}
\xdefinecolor{purpleish}{cmyk}{0.75,0.75,0,0}

\setbeamerfont{normal text}{size=\tiny}
%\setbeamerfont{framesubtitle}{size=\Large}
%\setbeamercolor{section in head/foot}{fg=white, bg=darkblue}
%\setbeamercolor{framesubtitle}{fg=darkblue, bg=white}

% can also choose different themes for the "inside" and "outside"

% \usepackage{beamerinnertheme_______}
% inner themes include circles, default, inmargin, rectangles, rounded

% \usepackage{beamerouterthemesmoothbars}
% outer themes include default, infolines, miniframes, shadow, sidebar, smoothbars, smoothtree, split, tree

%\useoutertheme[subsection=false]{smoothbars}

% to have the same footer on all slides
%\setbeamertemplate{footline}[text line]{STUFF HERE!}
%\setbeamertemplate{footline}[text line]{} % makes the footer EMPTY

\makeatother
\setbeamertemplate{frametitle continuation}{[\insertcontinuationcount]}
\setbeamertemplate{headline}[5]{}
\setbeamertemplate{footline}
{
  \leavevmode%
  \hbox{%
  \begin{beamercolorbox}[wd=.4\paperwidth,ht=2.25ex,dp=1ex,center]{author in head/foot}%
    \usebeamerfont{author in head/foot}\insertshortauthor
  \end{beamercolorbox}%
  \begin{beamercolorbox}[wd=.6\paperwidth,ht=2.25ex,dp=1ex,center]{title in head/foot}%
    \usebeamerfont{title in head/foot}\insertshorttitle\hspace*{3em}
    
  \end{beamercolorbox}}%
  \vskip0pt%
}
\makeatletter


\usepackage{url}
\usepackage{algorithmicx}
\usepackage{algorithm}
\usepackage{algpseudocode}
\usepackage{amsmath}
\usepackage{amssymb}
\usepackage{listings}
\usepackage{graphicx}
\usepackage{wrapfig}
\usepackage[sharp]{easylist}
\usepackage{tikz}

\newcommand{\teta}[1][]{\ensuremath{\theta_{#1}}}
\newcommand{\Jtheta}{\ensuremath{J\left(\teta[1],\dots,\teta[n]\right)}}
\makeatletter
\@ifclassloaded{beamer} {\renewcommand{\ALG@beginalgorithmic}{\scriptsize}}
%{\@ifclassloaded{beamer} {\algrenewcommand{\alglinenumber}[1]{\scriptsize}}
\makeatother 
%\renewcommand{\ALG@beginalgorithmic}{\small}} { } }

\makeatletter
% This is the vertical rule that is inserted
\def\therule{\makebox[\algorithmicindent][l]{\hspace*{.5em}\vrule height .75\baselineskip depth .25\baselineskip}}%

\newtoks\therules% Contains rules
\therules={}% Start with empty token list
\def\appendto#1#2{\expandafter#1\expandafter{\the#1#2}}% Append to token list
\def\gobblefirst#1{% Remove (first) from token list
  #1\expandafter\expandafter\expandafter{\expandafter\@gobble\the#1}}%
\def\LState{\State\unskip\the\therules}% New line-state
\def\LStatex{\Statex\unskip\the\therules}% New line-state
\def\pushindent{\appendto\therules\therule}%
\def\popindent{\gobblefirst\therules}%
\def\printindent{\unskip\the\therules}%
\def\printandpush{\printindent\pushindent}%
\def\popandprint{\popindent\printindent}%

%      ***      DECLARED LOOPS      ***
% (from algpseudocode.sty)
\algdef{SE}[WHILE]{While}{EndWhile}[1]
  {\printandpush\algorithmicwhile\ #1\ \algorithmicdo}
  {\popandprint\algorithmicend\ \algorithmicwhile}%
\algdef{SE}[FOR]{For}{EndFor}[1]
  {\printandpush\algorithmicfor\ #1\ \algorithmicdo}
  {\popandprint\algorithmicend\ \algorithmicfor}%
\algdef{S}[FOR]{ForAll}[1]
  {\printindent\algorithmicforall\ #1\ \algorithmicdo}%
\algdef{SE}[LOOP]{Loop}{EndLoop}
  {\printandpush\algorithmicloop}
  {\popandprint\algorithmicend\ \algorithmicloop}%
\algdef{SE}[REPEAT]{Repeat}{Until}
  {\printandpush\algorithmicrepeat}[1]
  {\popandprint\algorithmicuntil\ #1}%
\algdef{SE}[IF]{If}{EndIf}[1]
  {\printandpush\algorithmicif\ #1\ \algorithmicthen}
  {\popandprint\algorithmicend\ \algorithmicif}%
\algdef{C}[IF]{IF}{ElsIf}[1]
  {\popandprint\pushindent\algorithmicelse\ \algorithmicif\ #1\ \algorithmicthen}%
\algdef{Ce}[ELSE]{IF}{Else}{EndIf}
  {\popandprint\pushindent\algorithmicelse}%
\algdef{SE}[PROCEDURE]{Procedure}{EndProcedure}[2]
   {\printandpush\algorithmicprocedure\ \textproc{#1}\ifthenelse{\equal{#2}{}}{}{(#2)}}%
   {\popandprint\algorithmicend\ \algorithmicprocedure}%
\algdef{SE}[FUNCTION]{Function}{EndFunction}[2]
   {\printandpush\algorithmicfunction\ \textproc{#1}\ifthenelse{\equal{#2}{}}{}{(#2)}}%
   {\popandprint\algorithmicend\ \algorithmicfunction}%

\makeatother

\graphicspath{ {images/} }

\usepackage{subfigure}
\usepackage{multicol}
\usepackage[backend=bibtex]{biblatex}
%\bibliographystyle{unsrt}
%\bibliography{all}
%\bibliography{all}
%\addbibresource{prez.bib}
%\usepackage{epsfig}
%\usepackage{graphicx}
%\usepackage[all,knot]{xy}
%\xyoption{arc}
%\usepackage{multimedia}
%\usepackage{hyperref}

%\beamerdefaultoverlayspecification{<+->}

%%%%%%%%%%%%%%%%%%%%%%%%%%%%%%%%%%%%%%%%%%%%%%%%%%%%%%%%%%%%%%%%%%%%%%%%%%%%%%%%%%%%%%%%%%
%%%%%%%%%%%%%%%%%%%%%%%%%%%%%% Title Page Info %%%%%%%%%%%%%%%%%%%%%%%%%%%%%%%%%%%%%%%%%%%
%%%%%%%%%%%%%%%%%%%%%%%%%%%%%%%%%%%%%%%%%%%%%%%%%%%%%%%%%%%%%%%%%%%%%%%%%%%%%%%%%%%%%%%%%%
\setbeamercolor{title}{fg=white,bg=darkblue} 
\title[XSEDE]{XSEDE \\ e\textbf{X}treme \textbf{S}cience and \textbf{E}ngineering \textbf{D}iscovery \textbf{E}nvironment }

\subtitle[]{}
\author[Armando Fandango]{\textbf{Armando Fandango}\\UCF Campus Student Champion for XSEDE\\\textbf{Dr. Paul Wiegand}\\UCF Campus Champion for XSEDE}
\institute[University of Central Florida]{Advanced Research Computing Center \\ University of Central Florida }
\date{}

\footnotetext[3]{Visulation is a portmanteau describing a coupled system where graphic visualization and computer simulation occur simultaneously} 
%\titlegraphic{\includegraphics[width=0.5\textwidth]{hubble_25_anniversary.jpg}}
%%%%%%%%%%%%%%%%%%%%%%%%%%%%%%%%%%%%%%%%%%%%%%%%%%%%%%%%%%%%%%%%%%%%%%%%%%%%%%%%%%%%%%%%%%
%%%%%%%%%%%%%%%%%%%%%%%%%%%%%% Begin Your Document %%%%%%%%%%%%%%%%%%%%%%%%%%%%%%%%%%%%%%%
%%%%%%%%%%%%%%%%%%%%%%%%%%%%%%%%%%%%%%

\begin{document}

%%%%%%%%%%%%%%%%%%%%%%%%%%%%%%%

\frame[plain]{
	\titlepage 
}

%%%%%%%%%%%%%%%%%%%%%%%%%%%%%%%%%
\frame[plain]{\frametitle{Contents}
\tableofcontents
}

%%%%%%%%%%%%%%%%%%%%%%%%%%%%%%%%

%\section{Prologue}
%\frame{\tableofcontents[currentsection]}


\section{Introduction}

\begin{frame}{What is XSEDE?}
\pause
\centering
 from \url{www.XSEDE.org}
 
\begin{itemize}
\item XSEDE is the most advanced, powerful and robust collection of integrated digital resources and services in the world.
\item It is a single virtual computing system that scientists can use to interactively share resources, data and expertise.
\end{itemize}

\bigskip
\pause
According to what we know:
\bigskip
\pause

In nutshell, XSEDE is portal or gateway to \textbf{16} Supercomputer/HPC/HTC resources for Scientific Computing and Scientific Visualization, funded by US Agencies such as NSF, and available to US Researchers and Faculty. 
\bigskip
\bigskip




\end{frame}
\begin{frame}{What Resources are these ?}
from www.xsede.org/web/guest/resources/overview
\begin{itemize}
\item Compute Resources
\begin{itemize}

\item TACC: Stampede (XeonPhi)[102400 cores], Wrangler, Jetstream
\item PSC: Greenfield, Bridges Larg mem and Regular mem
\item SDSC: Gordon, Gordon ION. Comet [47616 cores]
\item LSU: SuperMIC 
\item NIC: Darter (Cray XC30) [23168 cores]
\item IU: Mason
\end{itemize}

\pause

\item Visualization Resources

\begin{itemize}
\item TACC: Maverick (NVIDIA GPGPU) [132 nodes, 2640 cores]
\end{itemize}

\pause

\item Storage Resources

\begin{itemize}
\item TACC: Ranch, Wrangler, XWFS, Jetstream
\item NICS: HPSS
\item PSC: Data Supercell, Bridges Pylon
\item SDSC: Data OASIS
\end{itemize}

\pause

\item HTC Resources

\begin{itemize}
\item USC: Open Science Grid - Consortium of 100+ Grids
\end{itemize}

\end{itemize}

\end{frame}


\section{Allocation}



\begin{frame}{I want one !!!!!! how ?}
\begin{itemize}
\pause
\item Create an account on XSEDE User Portal :  \url{portal.xsede.org/}
\pause
\item Write a startup, education or research allocation request - more on next slide
\pause
\item Apply for an allocation: \url{portal.xsede.org/submit-request/}
\end{itemize}
\end{frame}

\begin{frame}{What are the allocation types? and \\What do I need to write a request?}
\begin{itemize}
\pause
\item Startup 
\begin{itemize}
\pause
\item Apply with a CV and Abstract
\pause
\item Limited to some resources - check on XSEDE portal
\pause
\item Apply any time of the year

\end{itemize}
\pause
\item Education
\begin{itemize}
\pause
\item Apply with a CV, abstract, syllabus, student project info - simple
\pause
\item Limited to some resources, Apply any time of the year
\end{itemize}
\pause
\item Research
\begin{itemize}
\pause
\item Application process more involved
\pause
\item Summarize research in context of the current state of the art
\pause
\item Outline computational algorithms and their use in research
\pause
\item Justify need of computational resources - reviewers generally look for weak spots to reduce
\pause
\item Report results from previous research (XSEDE/non-XSEDE) - papers, benchmarks, code etc.
\end{itemize}
\end{itemize}
\end{frame}

\begin{frame}{I am a grad student.. Can I apply for allocation?}
\pause

Sorry... No.
\bigskip
\pause
Only a PI can submit an allocation proposal.

\bigskip
  
\begin{itemize}
\pause
\item Who can NOT be a PI
\pause
         \begin{itemize}
           \item students : high school, undergrad or grad (There are exceptions)
           
         \end{itemize}
 
 \pause
    \item Who can be a PI
      \begin{itemize}
      \pause
        \item A researcher or educator at a U.S. academic or non-profit research institution
        \pause
        \item A post-doctoral researcher
        \pause
        \item NSF Grad Student Fellows
        \pause
        \item A high school/K-12 Educator (education allocation)
      \end{itemize}
      
     
 
    \bigskip
    \pause
    Checkout \url{www.xsede.org} for current policies and rules
 
  
\end{itemize}
\end{frame}


\begin{frame}{Can I get more than one allocation?}
\begin{itemize}
\pause
 \item An individual may be a PI on only one active XSEDE allocation request at a given time
  \begin{itemize}
  \pause
  \item Combine your multiple projects into one request
  \end{itemize}
  \pause
  \item There are exceptions :-)
  \pause
  \begin{itemize}
  \item A research request can be submitted even if you have active startup allocation
  \pause
  \item A research request in entirely different field of science can be submitted even if you have an active research allocation,\\
  \item XSEDE Reviewers decide if its entirely different field of research.
  \pause
  \item Any number of educational requests can be made.
  \end{itemize}
  
  \end{itemize}
\end{frame}

\begin{frame}{How are XSEDE resources tracked?}
\begin{itemize}
\pause
\item Compute Resources
\begin{itemize}
\pause
\item Compute resources are allocated and used in terms of \textbf{SU}
\item SU stands for Service Units
\item 1 SU = 1 processor-core-hour

\pause
\item However some sites have more complex formulae that include factors such as queue priority.
\end{itemize}

\pause
\item Storage Resources
\begin{itemize}
\item Storage resources are allocated and used in terms of GB
\item GB stands for GigaBytes
\end{itemize}

\end{itemize}

\end{frame}

\begin{frame}{Tell me more about allocation...}
\begin{itemize}
\pause

  \item Allocation is for 12 month period
\pause
  \item Any unused SUs are forfeited at the end of 12 month period
  \pause
  \item Allocation can be requested for renewal
  \pause
  \item Startup allocation is not renewed. Go for research allocation after consuming your startup allocation.
  \pause
  \item PIs ask their team members to register their own free account on XSEDE and add them to allocation.
  \pause
  \item Additional resources for an allocation may be requested through a Supplement Request

 \end{itemize}
\end{frame}






\begin{frame}{Can I transfer SUs from one cluster to another ?}
\begin{itemize}
\item SUs can be transferred from one facility to another
\item However transferred SUs are based on a formula derived from HPL benchmark performances
\item Use SU Conversion Calculator : \url{www.xsede.org/web/guest/su-converter}

\end{itemize}

\end{frame}

\begin{frame}{So what software is available on these resources?}

\begin{itemize}
\pause
\item You can search here: \url{https://portal.xsede.org/software\#/} \newline [Demo] 
\pause
\item You need to check each and individual sites for:
\begin{itemize}
 \item What software is available and under which license?
 \item Can we install our own software and/or license?
\end{itemize}

\pause
\item Most of the sites use SLURM or TORQUE/MOAB
\begin{itemize}
 \item You may need to modify submission scripts and/or code
\end{itemize}

\pause
\item You can transfer large data using GridFTP/Globus
\begin{itemize}
 \item Transfer directly from one site to another site
 \item Single Sign On among other benefits
 \item Checkout for more info: \url{www.xsede.org/web/guest/data-transfers}
\pause
\item Stokes at ARCC UCF has GridFTP Capability, you can move large data to/from XSEDE clusters over High Speed Research Network - To Be Announced Soon
\end{itemize}

\end{itemize}
\end{frame}

\section{Prologue}

\begin{frame}{What are other XSEDE-like National Resources?}
\begin{itemize}

\pause
\item National Center for Supercomputing Applications (NCSA)
\begin{itemize}
\item NCSA Blue Waters 
\item 22,640 Cray XE6 nodes with 64 GB RAM and 4,228 Cray XK7 nodes with 32GB RAM, all with NVIDIA GPGPU
\item Lustre based Online Storage of 24+ PB with 1 TB/s r/w rate
\item www.ncsa.illinois.edu
\end{itemize}

\pause
\item Dept. of Energy: Advanced Scientific Computing Research (ASCR)
\begin{itemize}
\item \url{science.energy.gov/ascr/}
\end{itemize}

\pause
\item National Energy Research Scientific Computing Center (NERSC)
\begin{itemize}
\item\url{www.nersc.gov/users/computational-systems/}
\end{itemize}

\end{itemize}
\end{frame}



\begin{frame}{Whom do I contact at UCF regarding XSEDE?}
\begin{itemize}
\item Go to \url{www.xsede.org} first :-)
\pause
\item If you cant find what you need then get in touch with us
\bigskip
\item Dr. Paul Wiegand : wiegand@ist.ucf.edu
\item UCF Campus Champion for XSEDE
\bigskip
\item Armando Fandango: armando@ucf.edu
\item UCF Campus Student Champion for XSEDE
\end{itemize}

\end{frame}



\begin{frame}{Thank you and future directions}
\begin{itemize}
\item Would you like more detail about ...?
\begin{itemize}
\item XSEDE Science Gateways
\item Writing XSEDE Research Requests
\item SLURM on XSEDE (and Stokes) :-)
\item ..... Please write on the sheet being circulated
\end{itemize}
\pause
\item Thank you for your time and attention
\end{itemize}



\end{frame}

\begin{frame}[plain]
\begin{center}
\color{darkblue}
\huge
Thank You\\
\bigskip
\end{center} 
\end{frame}


%%%%%%%%%%%%%%%%%%%%%%%%%%%%%%%%%%%%%%%%%%%%%%%%%%%%%%%%%%%%%%%%%%%%%%%%%%%%%%%%%%%%%%%%%%
%%%%%%%%%%%%%%%%%%%%%%%%%%%%%% End Your Document %%%%%%%%%%%%%%%%%%%%%%%%%%%%%%%%%%%%%%%%%
%%%%%%%%%%%%%%%%%%%%%%%%%%%%%%%%%%%%%%%%%%%%%%%%%%%%%%%%%%%%%%%%%%%%%%%%%%%%%%%%%%%%%%%%%%

\end{document}

